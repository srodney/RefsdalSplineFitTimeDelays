\section{Photometry}\label{sec:Photometry}

For our photometric measurements on the difference images, we used the {\tt PythonPhot}\footnote{\url{https://github.com/djones1040/PythonPhot}} software package \citep{Jones:2015}, implemented using HST-specific wrapper functions from {\tt hstphot}\footnote{\url{https://github.com/srodney/hstphot}}, which was developed in part for use on other high-$z$ SNe observed with HST \citep[e.g.][]{Rodney:2015a,Rodney:2015b}.   We collected aperture photometry using 0.15\arcsec radius apertures, and a very local sky annulus with inner and outer radii of 0.5\arcsec and 1.0\arcsec.  This sky annulus was sized to fit between each Refsdal source and the region of the difference image contaminated by residuals from the lensing elliptical galaxy.  In addition, we measured the flux using a point spread function (PSF) fitting procedure similar to the DAOPHOT algorithm \citep{Stetson:1987}.  As in \citet{Rodney:2015a} and \citet{Rodney:2015b}, we used an empirical PSF model generated from HST imaging of the G2V standard star P330E, observed in a series of HST calibration programs.\footnote{PIs Deustua \& Kalirai, HST-PIDs 11451, 11926, 12334, 13089, and 13575}

To estimate photometric uncertainties in each image we planted 500 fake stars (copies of the model PSF) at random locations in the region defined by the sky annulus.  We measured the flux density of each fake star with PSF fitting and fit a normal distribution to the histogram of recovered fake star flux densities. A similar process was used for aperture flux uncertainties, except that we used empty apertures planted within the sky annulus on the difference images, and therefore measured the normal distribution about a mean near zero.   We then define two components of the uncertainty from the best-fit normal distribution. First, $\delta f_{\mu}$ is the difference between the measured mean of the distribution and the value of the flux assigned to all the planted fake stars, which is typically very close to zero but can gives an estimate of systematic biases in cases where the sky region around the SN is strongly contaminated by diffraction spikes or residuals from the lensing galaxy.  Second,  $\delta f_{\sigma}$ is the standard deviation of the best-fit normal distribution, and gives an empirical measure of the uncertainty due to sky noise and detector read noise. A final uncertainty component is $\delta f_{\nu}$, the Poisson noise error, computed from the total count of photons measured in the PSF fit or the aperture.  These are added in quadrature to give the total uncertainty: $\delta f$ = $\sqrt{ \delta f_{\mu}^2 + \delta f_{\sigma}^2 + \delta f_{\nu}^2}$.

These photometric measurements are reported in Table~\ref{tab:Photometry}, giving flux densities in micro-Janskys ($f_{\mu\rm{Jy}}$) and AB magnitudes.\footnote{Using $m_{\rm AB}=-2.5\log_{10}(f_{\mu\rm{Jy}})+23.9$}.  We mark with an asterisk any photometric measurement that was collected from despiked images prepared as in Figure~\ref{fig:Despiking}. Figure~\ref{fig:LightCurves} shows the resulting multi-band light curves for images S1-S4. The photometry measured using PSF fitting has better precision than the aperture photometry, but both sets are consistent within the errors.   

As a cross-check for systematic biases introduced by our primary data processing pipeline, we also processed all the HST images using two other data processing pipelines that make slightly different choices, for example in the number of images per epoch, the final pixel scale, and the {\tt AstroDrizzle} ``pixfrac'' parameter.  Aperture photometry on these images was found to be consistent within the errors. We found no significant features in the light curves that were present in one set of image products but not the others, so we infer that the subtle deviations from smoothness in the light curve shape can not be attributed to choices in the data processing. 
