\section{Discussion}\label{sec:Discussion}

Brief description of the model comparison project and results and reference to Treu et al. 2015

Brief discussion of the classification, and reference to Strolger et al 2015 and Kelly et al 2015.

Brief discussion of microlensing and millilensing: 
* discuss features in the light curves that could be attributed to micro/milli-lensing

* using multiple bands would be important for discriminating systematic errors from true m-lensing. 

Possible improvements to the spline-fitting approach with pycs for SN Refsdal time delay measurement :

* incorporate a slowly-varying color term, to allow additional photometric bands to be included, even if sparsely sampled. 

* splines are "too flexible", so incorporate SN template light curves (as long as the SN is not too peculiar)

What level of precision could be achieved with spline-fitting of a Type Ia or a normal CC SN? 

  * make some mock data using a SALT2 and CC template.

  * measure time delays with pycs

  * how finely sampled and how high S/N does the light curve need to be (in the absence of microlensing) to get time delay precisions on the order of a few days ?

  * (maybe: add in microlensing following Dobler \& Keeton 2006 and repeat) 




